% Vorlage für eine Bachelorarbeit - 2012 Timo Bingmann

% Dies ist nur eine Vorlage. Strikte Vorgaben wie die Bachelorarbeit auszusehen
% hat gibt es nicht. Darum können auch alle Teile angepasst werden.

\documentclass[12pt,a4paper,twoside]{scrartcl}

% Diese (und weitere) Eingabedateien sind in UTF-8
\usepackage[utf8]{inputenc}

% Verwende gute Type 1 Font: Latin Modern
\usepackage[T1]{fontenc}
\usepackage{lmodern}

% Sprache des Dokuments (für Silbentrennung und mehr)
\usepackage[german,english]{babel}

% Seitengröße - verwende fast die ganze A4 Seite
\usepackage[tmargin=22mm,bmargin=22mm,lmargin=20mm,rmargin=20mm]{geometry}

% Einrückung und Abstand zwischen Paragraphen
\setlength\parskip{\smallskipamount}
\setlength\parindent{0pt}

% Einige Standard-Mathematik Pakete
\usepackage{latexsym,amsmath,amssymb,mathtools,textcomp}

% Unterstützung für Sätze und Definitionen
\usepackage{amsthm}

\newtheorem{Satz}{Satz}[section]
\newtheorem{Definition}[Satz]{Definition}
\newtheorem{Lemma}[Satz]{Lemma}

\numberwithin{equation}{section}

% Deutsches Literaturverzeichnis
\usepackage{bibgerm}

% Unterstützung zum Einbinden von Graphiken
\usepackage{graphicx}

% Pakete die tabular und array verbessern
\usepackage{array,multirow}

% Kleiner enumerate und itemize Umgebungen
\usepackage{enumitem}

\setlist[enumerate]{topsep=0pt}
\setlist[itemize]{topsep=0pt}
\setlist[description]{font=\normalfont,topsep=0pt}

\setlist[enumerate,1]{label=(\roman*)}

% TikZ für Graphiken in LaTeX
\usepackage{tikz}
\usetikzlibrary{calc}

% Aktuelle Section und Untersection am Seitenkopf
\usepackage{fancyhdr}

\fancypagestyle{plain}{
  \fancyhead{}
  \fancyfoot{}
  \fancyfoot[LE,RO]{\normalsize\thepage}
  \renewcommand{\headrulewidth}{0pt}
  \renewcommand{\footrulewidth}{0pt}
}

\fancypagestyle{normal}{
  \setlength{\headheight}{20pt}
  \setlength\footskip{32pt}
  \fancyhead{}
  \fancyhead[LE]{\normalsize\textsc{\nouppercase{\leftmark}}}
  \fancyhead[RO]{\normalsize\textsc{\nouppercase{\rightmark}}}
  \fancyfoot{}
  \fancyfoot[LE,RO]{\normalsize\thepage}
  \renewcommand{\headrulewidth}{0.4pt}
  \renewcommand{\footrulewidth}{0pt}
}

% Hyperref für Hyperlink und Sprungtexte
\usepackage{xcolor,hyperref}

\hypersetup{
  pdftitle={Hier den Titel der Arbeit},
  pdfauthor={Hier den Author der Arbeit},
  pdfsubject={Stichworte, weiteres Stichwort},
  colorlinks=true,
  pdfborder={0 0 0},
  bookmarksopen=true,
  bookmarksopenlevel=1,
  bookmarksnumbered=true,
  linkcolor=blue!60!black,
  %linkcolor=black,
  citecolor=blue!60!black,
  urlcolor=blue!60!black,
  filecolor=green!60!black,
  pdfpagemode=UseNone,
  unicode=true,
}

\begin{document}

%%%%%%%%%%%%%%%%%%%%%%%%%%%%%%%%%%%%%%%%%%%%%%%%%%%%%%%%%%%%%%%%%%%%%%

\pagestyle{empty} % keine Seitenzahlen

% Titelblatt der Arbeit
\begin{titlepage}

  \begin{center}\large

    \quad\includegraphics[height=17mm]{kit_logo_de.pdf} \hfill
    \includegraphics[height=20mm]{grouplogo-algo-blue.pdf}\quad\null

    \vfill

    Bachelorarbeit
    \vspace*{2cm}

    {\bf\huge Titel der Bachelorarbeit}
    % Siehe auch oben die Felder pdftitle={}

    \vfill

    Name des Autors

    \vspace*{15mm}

    Abgabedatum: 19.10.2012

    \vspace*{45mm}

    \begin{tabular}{rl}
      Betreuer: & Prof. Dr. Peter Sanders \\
      & Dipl. Inform. Zweiter Betreuer \\
    \end{tabular}
    
    \vspace*{10mm}

    Institut für Theoretische Informatik, Algorithmik II \\
    Fakultät für Informatik \\
    Karlsruher Institut für Technologie

    \vspace*{12mm}
  \end{center}

\end{titlepage}

%%%%%%%%%%%%%%%%%%%%%%%%%%%%%%%%%%%%%%%%%%%%%%%%%%%%%%%%%%%%%%%%%%%%%%

\vspace*{0pt}\vfill

\hrule\medskip

Hiermit versichere ich, dass ich diese Arbeit selbständig verfasst und keine anderen, als die angegebenen Quellen und Hilfsmittel benutzt, die wörtlich oder inhaltlich übernommenen Stellen als solche kenntlich gemacht und die Satzung des Karlsruher Instituts für Technologie zur Sicherung guter wissenschaftlicher Praxis in der jeweils gültigen Fassung beachtet habe.

\bigskip

\noindent
Ort, den Datum

% Unterschrift (handgeschrieben)

\vspace*{5cm}

\clearpage

%%%%%%%%%%%%%%%%%%%%%%%%%%%%%%%%%%%%%%%%%%%%%%%%%%%%%%%%%%%%%%%%%%%%%%

\vspace*{0pt}\vfill

\selectlanguage{german}
\begin{abstract}
\centerline{\bf Zusammenfassung}

Hier die deutsche Zusammenfassung.

Ich bin Blindtext. Von Geburt an. Es hat lange gedauert, bis ich begriffen habe, was es bedeutet, ein blinder Text zu sein: Man macht keinen Sinn. Man wirkt hier und da aus dem Zusammenhang gerissen. Oft wird man gar nicht erst gelesen. Aber bin ich deshalb ein schlechter Text? Ich weiß, dass ich nie die Chance haben werde im Stern zu erscheinen. Aber bin ich darum weniger wichtig? Ich bin blind! Aber ich bin gerne Text. Und sollten Sie mich jetzt tatsächlich zu Ende lesen, dann habe ich etwas geschafft, was den meisten ,,normalen`` Texten nicht gelingt.

Ich bin Blindtext. Von Geburt an. Es hat lange gedauert, bis ich begriffen habe, was es bedeutet, ein blinder Text zu sein: Man macht keinen Sinn. Man wirkt hier und da aus dem Zusammenhang gerissen. Oft wird man gar nicht erst gelesen. Aber bin ich deshalb ein schlechter Text? Ich weiß, dass ich nie die Chance haben werde im Stern zu erscheinen. Aber bin ich darum weniger wichtig? Ich bin blind! Aber ich bin gerne Text.

\end{abstract}

\vfill

\selectlanguage{english}
\begin{abstract}
\centerline{\bf Abstract}

And here an English translation of the German abstract.

I'm blind text. From birth. It took a long time until I realized what it means to be random text: You make no sense. You stand here and there out of context. Frequently, they do not even read. But I have a bad copy? I know that I will never have the chance of appearing in the. But I'm any less important? I'm blind! But I like to text. And you should see me now actually over, then I have accomplished something that is not possible in most ``normal'' copies.

I'm blind text. From birth. It took a long time until I realized what it means to be random text: You make no sense. You stand here and there out of context. Frequently, they do not even read. But I have a bad copy? I know that I will never have the chance of appearing in the. But I'm any less important? I'm blind! But I like to text.

\end{abstract}
\selectlanguage{german}

\vfill\vfill\vfill
\clearpage

%%%%%%%%%%%%%%%%%%%%%%%%%%%%%%%%%%%%%%%%%%%%%%%%%%%%%%%%%%%%%%%%%%%%%%

\vspace*{0pt}\vfill

\section*{Danksagungen}

Lorem ipsum dolor sit amet, consectetuer adipiscing elit. Aenean commodo ligula eget dolor. Aenean massa. Cum sociis natoque penatibus et magnis dis parturient montes, nascetur ridiculus mus. Donec quam felis, ultricies nec, pellentesque eu, pretium quis, sem. Nulla consequat massa quis enim. Donec pede justo, fringilla vel, aliquet nec, vulputate eget, arcu. In enim justo, rhoncus ut, imperdiet a, venenatis vitae, justo.

\vfill\vfill\vfill
\clearpage

%%%%%%%%%%%%%%%%%%%%%%%%%%%%%%%%%%%%%%%%%%%%%%%%%%%%%%%%%%%%%%%%%%%%%%

\pagestyle{normal}

% Inhaltsverzeichnis
\tableofcontents

\clearpage

%%%%%%%%%%%%%%%%%%%%%%%%%%%%%%%%%%%%%%%%%%%%%%%%%%%%%%%%%%%%%%%%%%%%%%

\listoffigures
\listoftables

\clearpage

%%%%%%%%%%%%%%%%%%%%%%%%%%%%%%%%%%%%%%%%%%%%%%%%%%%%%%%%%%%%%%%%%%%%%%

\section{Einleitung}

\subsection{Motivation}

Weit hinten, hinter den Wortbergen, fern der Länder Vokalien und Konsonantien leben die Blindtexte. Abgeschieden wohnen Sie in Buchstabhausen an der Küste des Semantik, eines großen Sprachozeans. Ein kleines Bächlein namens Duden fließt durch ihren Ort und versorgt sie mit den nötigen Regelialien. Es ist ein paradiesmatisches Land, in dem einem gebratene Satzteile in den Mund fliegen. Nicht einmal von der allmächtigen Interpunktion werden die Blindtexte beherrscht – ein geradezu unorthographisches Leben. Eines Tages aber beschloß eine kleine Zeile Blindtext, ihr Name war Lorem Ipsum, hinaus zu gehen in die weite Grammatik. Der große Oxmox riet ihr davon ab, da es dort wimmele von bösen Kommata, wilden Fragezeichen und hinterhältigen Semikoli, doch das Blindtextchen ließ sich nicht beirren. Es packte seine sieben Versalien, schob sich sein Initial in den Gürtel und machte sich auf den Weg. Als es die ersten Hügel des Kursivgebirges erklommen hatte, warf es einen letzten Blick zurück auf die Skyline seiner Heimatstadt Buchstabhausen, die Headline von Alphabetdorf und die Subline seiner eigenen Straße, der Zeilengasse.

Wehmütig lief ihm eine rhetorische Frage über die Wange, dann setzte es seinen Weg fort. Unterwegs traf es eine Copy. Die Copy warnte das Blindtextchen, da, wo sie herkäme wäre sie zigmal umgeschrieben worden und alles, was von ihrem Ursprung noch übrig wäre, sei das Wort ,,und`` und das Blindtextchen solle umkehren und wieder in sein eigenes, sicheres Land zurückkehren. Doch alles Gutzureden konnte es nicht überzeugen und so dauerte es nicht lange, bis ihm ein paar heimtückische Werbetexter auflauerten, es mit Longe und Parole betrunken machten und es dann in ihre Agentur schleppten, wo sie es für ihre Projekte wieder und wieder mißbrauchten. Und wenn es nicht umgeschrieben wurde, dann benutzen Sie es immernoch.
\begin{enumerate}
\item Weit hinten, hinter den Wortbergen, fern der Länder Vokalien und Konsonantien leben die Blindtexte.
\item Abgeschieden wohnen Sie in Buchstabhausen an der Küste des Semantik, eines großen Sprachozeans.
\item
Ein kleines Bächlein namens Duden fließt durch ihren Ort und versorgt sie mit den nötigen Regelialien. Es ist ein paradiesmatisches Land, in dem einem gebratene Satzteile in den Mund fliegen.
\end{enumerate}
Nicht einmal von der allmächtigen Interpunktion werden die Blindtexte beherrscht – ein geradezu unorthographisches Leben. Eines Tages aber beschloß eine kleine Zeile Blindtext, ihr Name war Lorem Ipsum, hinaus zu gehen in die weite Grammatik. Der große Oxmox riet ihr davon ab, da es dort wimmele von bösen Kommata, wilden Fragezeichen und hinterhältigen Semikoli, doch das Blindtextchen ließ sich nicht beirren. Es packte seine sieben Versalien, schob sich sein Initial in den Gürtel und machte sich auf den Weg. Als es die ersten Hügel des Kursivgebirges erklommen hatte, warf es einen letzten Blick zurück auf die Skyline seiner Heimatstadt Buchstabhausen, die Headline von Alphabetdorf und die Subline seiner eigenen Straße, der Zeilengasse. Wehmütig lief ihm eine rhetorische Frage über die Wange, dann setzte es seinen Weg fort. Unterwegs traf es eine Copy. Die Copy warnte das Blindtextchen, da, wo sie herkäme wäre sie zigmal umgeschrieben worden und alles, was von ihrem Ursprung noch übrig wäre, sei das Wort ,,und`` und das Blindtextchen solle umkehren und wieder in sein eigenes, sicheres Land zurückkehren.

\subsection{Beiträge der Bachelorarbeit}

Doch alles Gutzureden konnte es nicht überzeugen und so dauerte es nicht lange, bis ihm ein paar heimtückische Werbetexter auflauerten, es mit Longe und Parole betrunken machten und es dann in ihre Agentur schleppten, wo sie es für ihre Projekte wieder und wieder mißbrauchten. Und wenn es nicht umgeschrieben wurde, dann benutzen Sie es immernoch. Weit hinten, hinter den Wortbergen, fern der Länder Vokalien und Konsonantien leben die Blindtexte. Abgeschieden wohnen Sie in Buchstabhausen an der Küste des Semantik, eines großen Sprachozeans. Ein kleines Bächlein namens Duden fließt durch ihren Ort und versorgt sie mit den nötigen Regelialien. Es ist ein paradiesmatisches Land, in dem einem gebratene Satzteile in den Mund fliegen. Nicht einmal von der allmächtigen Interpunktion werden die Blindtexte beherrscht – ein geradezu unorthographisches Leben. Eines Tages aber beschloß eine kleine Zeile Blindtext, ihr Name war Lorem Ipsum, hinaus zu gehen in die weite Grammatik. Der große Oxmox riet ihr davon ab, da es dort wimmele von bösen Kommata, wilden Fragezeichen und hinterhältigen Semikoli, doch das Blindtextchen ließ sich nicht beirren.

%%%%%%%%%%%%%%%%%%%%%%%%%%%%%%%%%%%%%%%%%%%%%%%%%%%%%%%%%%%%%%%%%%%%%%

\section{Bekannte Algorithmen}

Es packte seine sieben Versalien \cite{Cormen2001introduction}, schob sich sein Initial in den Gürtel und machte sich auf den Weg. Als es die ersten Hügel des Kursivgebirges erklommen hatte, warf es einen letzten Blick zurück auf die Skyline seiner Heimatstadt Buchstabhausen, die Headline von Alphabetdorf und die Subline seiner eigenen Straße, der Zeilengasse. Wehmütig lief ihm eine rhetorische Frage über die Wange, dann setzte es seinen Weg fort. Unterwegs traf es eine Copy.
\begin{Definition}
Die Copy warnte das Blindtextchen, da, wo sie herkäme wäre sie zigmal umgeschrieben worden und alles, was von ihrem Ursprung noch übrig wäre, sei das Wort ,,und`` und das Blindtextchen solle umkehren und wieder in sein eigenes, sicheres Land zurückkehren.
\end{Definition}
Doch alles Gutzureden konnte es nicht überzeugen und so dauerte es nicht lange, bis ihm ein paar heimtückische Werbetexter auflauerten, es mit Longe und Parole betrunken machten und es dann in ihre Agentur schleppten, wo sie es für ihre Projekte wieder und wieder mißbrauchten. Und wenn es nicht umgeschrieben wurde, dann benutzen Sie es immernoch. Weit hinten, hinter den Wortbergen, fern der Länder Vokalien und Konsonantien leben die Blindtexte.

\subsection{Enumeration aller Möglichkeiten}

Abgeschieden wohnen Sie in Buchstabhausen an der Küste des Semantik, eines großen Sprachozeans. Ein kleines Bächlein namens Duden fließt durch ihren Ort und versorgt sie mit den nötigen Regelialien. Es ist ein paradiesmatisches Land, in dem einem gebratene Satzteile in den Mund fliegen. Nicht einmal von der allmächtigen Interpunktion werden die Blindtexte beherrscht – ein geradezu unorthographisches Leben. Eines Tages aber beschloß eine kleine Zeile Blindtext, ihr Name war Lorem Ipsum, hinaus zu gehen in die weite Grammatik.

\subsection{Formulierung als lineares Programm}

Der große Oxmox riet ihr davon ab, da es dort wimmele von bösen Kommata, wilden Fragezeichen und hinterhältigen Semikoli, doch das Blindtextchen ließ sich nicht beirren. Es packte seine sieben Versalien, schob sich sein Initial in den Gürtel und machte sich auf den Weg. Als es die ersten Hügel des Kursivgebirges erklommen hatte, warf es einen letzten Blick zurück auf die Skyline seiner Heimatstadt Buchstabhausen, die Headline von Alphabetdorf und die Subline seiner eigenen Straße, der Zeilengasse. Wehmütig lief ihm eine rhetorische Frage über die Wange, dann setzte es seinen Weg fort. Unterwegs traf es eine Copy:
$$\max \{ c x \mid x \in \mathbb{R}^n \,,\; x \geq 0 \,,\; A x \leq b \} \,.$$
Die Copy warnte das Blindtextchen, da, wo sie herkäme wäre sie zigmal umgeschrieben worden und alles, was von ihrem Ursprung noch übrig wäre, sei das Wort ,,und`` und das Blindtextchen solle umkehren und wieder in sein eigenes, sicheres Land zurückkehren. Doch alles Gutzureden konnte es nicht überzeugen und so dauerte es nicht lange, bis ihm ein paar heimtückische Werbetexter auflauerten, es mit Longe und Parole betrunken machten und es dann in ihre Agentur schleppten, wo sie es für ihre Projekte wieder und wieder mißbrauchten. Und wenn es nicht umgeschrieben wurde, dann benutzen Sie es immernoch.
\begin{Satz}
Weit hinten, hinter den Wortbergen, fern der Länder Vokalien und Konsonantien leben die Blindtexte.
\begin{proof}
Abgeschieden wohnen Sie in Buchstabhausen an der Küste des Semantik, eines großen Sprachozeans. Ein kleines Bächlein namens Duden fließt durch ihren Ort und versorgt sie mit den nötigen Regelialien. Es ist ein paradiesmatisches Land, in dem einem gebratene Satzteile in den Mund fliegen. Nicht einmal von der allmächtigen Interpunktion werden die Blindtexte beherrscht – ein geradezu unorthographisches Leben.
\end{proof}
\end{Satz}

%%%%%%%%%%%%%%%%%%%%%%%%%%%%%%%%%%%%%%%%%%%%%%%%%%%%%%%%%%%%%%%%%%%%%%

\section{Implementierung}

Eines Tages aber beschloß eine kleine Zeile Blindtext, ihr Name war Lorem Ipsum, hinaus zu gehen in die weite Grammatik. Der große Oxmox riet ihr davon ab, da es dort wimmele von bösen Kommata, wilden Fragezeichen und hinterhältigen Semikoli, doch das Blindtextchen ließ sich nicht beirren. Es packte seine sieben Versalien, schob sich sein Initial in den Gürtel und machte sich auf den Weg. Als es die ersten Hügel des Kursivgebirges erklommen hatte, warf es einen letzten Blick zurück auf die Skyline seiner Heimatstadt Buchstabhausen, die Headline von Alphabetdorf und die Subline seiner eigenen Straße, der Zeilengasse. Wehmütig lief ihm eine rhetorische Frage über die Wange, dann setzte es seinen Weg fort. Unterwegs traf es eine Copy. Die Copy warnte das Blindtextchen, da, wo sie herkäme wäre sie zigmal umgeschrieben worden und alles, was von ihrem Ursprung noch übrig wäre, sei das Wort ,,und`` und das Blindtextchen solle umkehren und wieder in sein eigenes, sicheres Land zurückkehren.

Doch alles Gutzureden konnte es nicht überzeugen und so dauerte es nicht lange, bis ihm ein paar heimtückische Werbetexter auflauerten, es mit Longe und Parole betrunken machten und es dann in ihre Agentur schleppten, wo sie es für ihre Projekte wieder und wieder mißbrauchten. Und wenn es nicht umgeschrieben wurde, dann benutzen Sie es immernoch. Weit hinten, hinter den Wortbergen, fern der Länder Vokalien und Konsonantien leben die Blindtexte. Abgeschieden wohnen Sie in Buchstabhausen an der Küste des Semantik, eines großen Sprachozeans. Ein kleines Bächlein namens Duden fließt durch ihren Ort und versorgt sie mit den nötigen Regelialien. Es ist ein paradiesmatisches Land, in dem einem gebratene Satzteile in den Mund fliegen (siehe \autoref{fig:binäre Suchbäume}). Nicht einmal von der allmächtigen Interpunktion werden die Blindtexte beherrscht – ein geradezu unorthographisches Leben. Eines Tages aber beschloß eine kleine Zeile Blindtext, ihr Name war Lorem Ipsum, hinaus zu gehen in die weite Grammatik. Der große Oxmox riet ihr davon ab, da es dort wimmele von bösen Kommata, wilden Fragezeichen und hinterhältigen Semikoli, doch das Blindtextchen ließ sich nicht beirren.

\begin{figure}
\hfill%
\begin{tikzpicture}[font=\small,
  odot/.style={draw,circle,inner sep=0pt,minimum width=18pt},
  ]
  \node[odot] (0) at (0, -0) {25};
  \node[odot] (1) at (-2, -1) {10};
  \node[odot] (2) at (2, -1) {31};
  \node[odot] (3) at (-3, -2) {4};
  \node[odot] (4) at (-1, -2) {21};
  \node[odot] (5) at (1, -2) {30};
  \node[odot] (6) at (3, -2) {32};
  \node[odot] (7) at (-3.5, -3) {2};
  \node[odot] (8) at (-2.5, -3) {8};
  \node[odot] (9) at (-1.5, -3) {11};
  \node[odot] (10) at (-0.5, -3) {24};
  \node[odot] (11) at (0.5, -3) {28};
  \draw (4) edge (10);
  \draw (3) edge (8);
  \draw (1) edge (3);
  \draw (0) edge (2);
  \draw (2) edge (6);
  \draw (4) edge (9);
  \draw (2) edge (5);
  \draw (1) edge (4);
  \draw (0) edge (1);
  \draw (3) edge (7);
  \draw (5) edge (11);
\end{tikzpicture}
\hfill%
\begin{tikzpicture}[font=\small,
  odot/.style={draw,circle,inner sep=0pt,minimum width=18pt},
  ]
  \node[odot] (0) at (0, -0) {21};
  \node[odot] (1) at (-2, -1) {8};
  \node[odot] (2) at (2, -1) {30};
  \node[odot] (3) at (-3, -2) {4};
  \node[odot] (4) at (-1, -2) {11};
  \node[odot] (5) at (1, -2) {25};
  \node[odot] (6) at (3, -2) {31};
  \node[odot] (7) at (-3.5, -3) {2};
  \node[odot] (8) at (-1.5, -3) {10};
  \node[odot] (9) at (0.5, -3) {24};
  \node[odot] (10) at (1.5, -3) {28};
  \node[odot] (11) at (3.5, -3) {32};
  \draw (5) edge (9);
  \draw (0) edge (2);
  \draw (1) edge (4);
  \draw (5) edge (10);
  \draw (2) edge (5);
  \draw (6) edge (11);
  \draw (0) edge (1);
  \draw (4) edge (8);
  \draw (3) edge (7);
  \draw (1) edge (3);
  \draw (2) edge (6);
\end{tikzpicture}
\hfill\null%

\caption{Zwei binäre Suchbäume für $\{ 2, 4, 8, 10, 11, 21, 24, 25, 28, 30, 31, 32 \}$}\label{fig:binäre Suchbäume}
\end{figure}

Es packte seine sieben Versalien, schob sich sein Initial in den Gürtel und machte sich auf den Weg. Als es die ersten Hügel des Kursivgebirges erklommen hatte, warf es einen letzten Blick zurück auf die Skyline seiner Heimatstadt Buchstabhausen, die Headline von Alphabetdorf und die Subline seiner eigenen Straße, der Zeilengasse. Wehmütig lief ihm eine rhetorische Frage über die Wange, dann setzte es seinen Weg fort. Unterwegs traf es eine Copy. Die Copy warnte das Blindtextchen, da, wo sie herkäme wäre sie zigmal umgeschrieben worden und alles, was von ihrem Ursprung noch übrig wäre, sei das Wort ,,und`` und das Blindtextchen solle umkehren und wieder in sein eigenes, sicheres Land zurückkehren. Doch alles Gutzureden konnte es nicht überzeugen und so dauerte es nicht lange, bis ihm ein paar heimtückische Werbetexter auflauerten, es mit Longe und Parole betrunken machten und es dann in ihre Agentur schleppten, wo sie es für ihre Projekte wieder und wieder mißbrauchten. Und wenn es nicht umgeschrieben wurde, dann benutzen Sie es immernoch. Weit hinten, hinter den Wortbergen, fern der Länder Vokalien und Konsonantien leben die Blindtexte.

Abgeschieden wohnen Sie in Buchstabhausen an der Küste des Semantik, eines großen Sprachozeans. Ein kleines Bächlein namens Duden fließt durch ihren Ort und versorgt sie mit den nötigen Regelialien. Es ist ein paradiesmatisches Land, in dem einem gebratene Satzteile in den Mund fliegen. Nicht einmal von der allmächtigen Interpunktion werden die Blindtexte beherrscht – ein geradezu unorthographisches Leben. Eines Tages aber beschloß eine kleine Zeile Blindtext, ihr Name war Lorem Ipsum, hinaus zu gehen in die weite Grammatik. Der große Oxmox riet ihr davon ab, da es dort wimmele von bösen Kommata, wilden Fragezeichen und hinterhältigen Semikoli, doch das Blindtextchen ließ sich nicht beirren. Es packte seine sieben Versalien, schob sich sein Initial in den Gürtel und machte sich auf den Weg. Als es die ersten Hügel des Kursivgebirges erklommen hatte, warf es einen letzten Blick zurück auf die Skyline seiner Heimatstadt Buchstabhausen, die Headline von Alphabetdorf und die Subline seiner eigenen Straße, der Zeilengasse. Wehmütig lief ihm eine rhetorische Frage über die Wange, dann setzte es seinen Weg fort. Unterwegs traf es eine Copy.

%%%%%%%%%%%%%%%%%%%%%%%%%%%%%%%%%%%%%%%%%%%%%%%%%%%%%%%%%%%%%%%%%%%%%%

\section{Experimentelle Ergebnisse}

\subsection{Laufzeit der Enumeration}

Die Copy warnte das Blindtextchen, da, wo sie herkäme wäre sie zigmal umgeschrieben worden und alles, was von ihrem Ursprung noch übrig wäre, sei das Wort ,,und`` und das Blindtextchen solle umkehren und wieder in sein eigenes, sicheres Land zurückkehren. Doch alles Gutzureden konnte es nicht überzeugen und so dauerte es nicht lange, bis ihm ein paar heimtückische Werbetexter auflauerten, es mit Longe und Parole betrunken machten und es dann in ihre Agentur schleppten, wo sie es für ihre Projekte wieder und wieder mißbrauchten. Und wenn es nicht umgeschrieben wurde, dann benutzen Sie es immernoch. Weit hinten, hinter den Wortbergen, fern der Länder Vokalien und Konsonantien leben die Blindtexte. Abgeschieden wohnen Sie in Buchstabhausen an der Küste des Semantik, eines großen Sprachozeans. Ein kleines Bächlein namens Duden fließt durch ihren Ort und versorgt sie mit den nötigen Regelialien. Es ist ein paradiesmatisches Land, in dem einem gebratene Satzteile in den Mund fliegen. Nicht einmal von der allmächtigen Interpunktion werden die Blindtexte beherrscht – ein geradezu unorthographisches Leben.

Eines Tages aber beschloß eine kleine Zeile Blindtext, ihr Name war Lorem Ipsum, hinaus zu gehen in die weite Grammatik. Der große Oxmox riet ihr davon ab, da es dort wimmele von bösen Kommata, wilden Fragezeichen und hinterhältigen Semikoli, doch das Blindtextchen ließ sich nicht beirren. Es packte seine sieben Versalien, schob sich sein Initial in den Gürtel und machte sich auf den Weg. Als es die ersten Hügel des Kursivgebirges erklommen hatte, warf es einen letzten Blick zurück auf die Skyline seiner Heimatstadt Buchstabhausen, die Headline von Alphabetdorf und die Subline seiner eigenen Straße, der Zeilengasse. Wehmütig lief ihm eine rhetorische Frage über die Wange, dann setzte es seinen Weg fort. Unterwegs traf es eine Copy. Die Copy warnte das Blindtextchen, da, wo sie herkäme wäre sie zigmal umgeschrieben worden und alles, was von ihrem Ursprung noch übrig wäre, sei das Wort ,,und`` und das Blindtextchen solle umkehren und wieder in sein eigenes, sicheres Land zurückkehren.

\subsection{Laufzeit mit CPLEX}

Doch alles Gutzureden konnte es nicht überzeugen und so dauerte es nicht lange, bis ihm ein paar heimtückische Werbetexter auflauerten, es mit Longe und Parole betrunken machten und es dann in ihre Agentur schleppten, wo sie es für ihre Projekte wieder und wieder mißbrauchten. Und wenn es nicht umgeschrieben wurde, dann benutzen Sie es immernoch. Weit hinten, hinter den Wortbergen, fern der Länder Vokalien und Konsonantien leben die Blindtexte. Abgeschieden wohnen Sie in Buchstabhausen an der Küste des Semantik, eines großen Sprachozeans. Ein kleines Bächlein namens Duden fließt durch ihren Ort und versorgt sie mit den nötigen Regelialien. Es ist ein paradiesmatisches Land, in dem einem gebratene Satzteile in den Mund fliegen. Nicht einmal von der allmächtigen Interpunktion werden die Blindtexte beherrscht – ein geradezu unorthographisches Leben. Eines Tages aber beschloß eine kleine Zeile Blindtext, ihr Name war Lorem Ipsum, hinaus zu gehen in die weite Grammatik. Der große Oxmox riet ihr davon ab, da es dort wimmele von bösen Kommata, wilden Fragezeichen und hinterhältigen Semikoli, doch das Blindtextchen ließ sich nicht beirren.

%%%%%%%%%%%%%%%%%%%%%%%%%%%%%%%%%%%%%%%%%%%%%%%%%%%%%%%%%%%%%%%%%%%%%%

\section{Schlussfolgerungen}

\subsection{Ergebnisse und Bewertung}

Es packte seine sieben Versalien, schob sich sein Initial in den Gürtel und machte sich auf den Weg. Als es die ersten Hügel des Kursivgebirges erklommen hatte, warf es einen letzten Blick zurück auf die Skyline seiner Heimatstadt Buchstabhausen, die Headline von Alphabetdorf und die Subline seiner eigenen Straße, der Zeilengasse. Wehmütig lief ihm eine rhetorische Frage über die Wange, dann setzte es seinen Weg fort. Unterwegs traf es eine Copy. Die Copy warnte das Blindtextchen, da, wo sie herkäme wäre sie zigmal umgeschrieben worden und alles, was von ihrem Ursprung noch übrig wäre, sei das Wort ,,und`` und das Blindtextchen solle umkehren und wieder in sein eigenes, sicheres Land zurückkehren. Doch alles Gutzureden konnte es nicht überzeugen und so dauerte es nicht lange, bis ihm ein paar heimtückische Werbetexter auflauerten, es mit Longe und Parole betrunken machten und es dann in ihre Agentur schleppten, wo sie es für ihre Projekte wieder und wieder mißbrauchten. Und wenn es nicht umgeschrieben wurde, dann benutzen Sie es immernoch. Weit hinten, hinter den Wortbergen, fern der Länder Vokalien und Konsonantien leben die Blindtexte.

Abgeschieden wohnen Sie in Buchstabhausen an der Küste des Semantik, eines großen Sprachozeans. Ein kleines Bächlein namens Duden fließt durch ihren Ort und versorgt sie mit den nötigen Regelialien. Es ist ein paradiesmatisches Land, in dem einem gebratene Satzteile in den Mund fliegen. Nicht einmal von der allmächtigen Interpunktion werden die Blindtexte beherrscht – ein geradezu unorthographisches Leben. Eines Tages aber beschloß eine kleine Zeile Blindtext, ihr Name war Lorem Ipsum, hinaus zu gehen in die weite Grammatik. Der große Oxmox riet ihr davon ab, da es dort wimmele von bösen Kommata, wilden Fragezeichen und hinterhältigen Semikoli, doch das Blindtextchen ließ sich nicht beirren. Es packte seine sieben Versalien, schob sich sein Initial in den Gürtel und machte sich auf den Weg. Als es die ersten Hügel des Kursivgebirges erklommen hatte, warf es einen letzten Blick zurück auf die Skyline seiner Heimatstadt Buchstabhausen, die Headline von Alphabetdorf und die Subline seiner eigenen Straße, der Zeilengasse. Wehmütig lief ihm eine rhetorische Frage über die Wange, dann setzte es seinen Weg fort. Unterwegs traf es eine Copy.

\subsection{Mögliche Erweiterungen}

Die Copy warnte das Blindtextchen, da, wo sie herkäme wäre sie zigmal umgeschrieben worden und alles, was von ihrem Ursprung noch übrig wäre, sei das Wort ,,und`` und das Blindtextchen solle umkehren und wieder in sein eigenes, sicheres Land zurückkehren. Doch alles Gutzureden konnte es nicht überzeugen und so dauerte es nicht lange, bis ihm ein paar heimtückische Werbetexter auflauerten, es mit Longe und Parole betrunken machten und es dann in ihre Agentur schleppten, wo sie es für ihre Projekte wieder und wieder mißbrauchten. Und wenn es nicht umgeschrieben wurde, dann benutzen Sie es immernoch. Weit hinten, hinter den Wortbergen, fern der Länder Vokalien und Konsonantien leben die Blindtexte. Abgeschieden wohnen Sie in Buchstabhausen an der Küste des Semantik, eines großen Sprachozeans. Ein kleines Bächlein namens Duden fließt durch ihren Ort und versorgt sie mit den nötigen Regelialien. Es ist ein paradiesmatisches Land, in dem einem gebratene Satzteile in den Mund fliegen. Nicht einmal von der allmächtigen Interpunktion werden die Blindtexte beherrscht – ein geradezu unorthographisches Leben.

\clearpage

%%%%%%%%%%%%%%%%%%%%%%%%%%%%%%%%%%%%%%%%%%%%%%%%%%%%%%%%%%%%%%%%%%%%%%

\bibliographystyle{gerplain}
\bibliography{literatur}

\end{document}
